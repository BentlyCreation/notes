\documentclass[a4paper, 11pt]{ctexart}

%===========================================================
% 宏包引言区
%===========================================================
\usepackage[left=2.5cm, right=2.5cm, top=2.5cm, bottom=2.5cm]{geometry} % 页面边距
\usepackage{amsmath, amssymb, amsfonts, bm} % 数学公式与粗体向量
\usepackage{booktabs} % 三线表
\usepackage{graphicx} % 图片支持
\usepackage{float}    % 图片浮动位置控制
\usepackage{hyperref} % 生成可点击的目录
\usepackage{xcolor}   % 颜色支持
\usepackage{listings} % 代码块支持
\usepackage{fancyhdr} % 页眉页脚
\usepackage{tcolorbox} % 文本框
% --- 伪代码专用宏包 ---
\usepackage{algorithm}
\usepackage{algpseudocode} 
\usepackage{amsmath} % 数学公式支持

% 汉化算法关键词(可选,如果你想显示“输入/输出”而不是“Input/Output”)
% \floatname{algorithm}{算法}
% \renewcommand{\algorithmicrequire}{\textbf{输入:}}
% \renewcommand{\algorithmicensure}{\textbf{输出:}}

% 或者保持英文高逼格(推荐)
\renewcommand{\algorithmicrequire}{\textbf{Input:}}
\renewcommand{\algorithmicensure}{\textbf{Output:}}
%===========================================================
% 页面设置
%===========================================================
\pagestyle{fancy}
\fancyhf{}
\fancyhead[L]{模式识别学习记录} % 左页眉
\fancyhead[R]{\thepage}        % 右页眉
\renewcommand{\headrulewidth}{0.5pt}

% 设置超链接颜色
\hypersetup{
    colorlinks=true,
    linkcolor=black,
    filecolor=magenta,      
    urlcolor=blue,
}

% 定义“重点”文本框
\newtcolorbox{important}[1][]{colback=red!5!white, colframe=red!75!black, title=考点/重点: #1}

%===========================================================
% 文档基本信息
%===========================================================
\title{\textbf{模式识别 (Pattern Recognition) \\ 期末复习笔记}}
\author{整理人:Your Name}
\date{\today}

%===========================================================
% 正文开始
%===========================================================
\begin{document}

\maketitle
\tableofcontents
\newpage % 目录后强制换页

%===========================================================
% 正文内容区
%===========================================================

\section{绪论 \small{(3学时 向世明)}}

\subsection{模式识别基础}
\begin{itemize}
    \item \textbf{基本定义}:
    \begin{itemize}
        \item \textbf{模式 (Pattern)}:对客体(Object)或事件(Event)的描述,通常是向量形式。
        \item \textbf{模式类 (Pattern Class)}:具有共同属性的模式集合。
        \item \textbf{模式识别}:利用机器(计算机)自动将特定模式映射到特定类别的过程。
    \end{itemize}
    \item \textbf{经典实例}:
    \begin{itemize}
        \item \textbf{硬币分类}:利用尺寸、重量等特征区分不同面值的硬币。
        \item \textbf{鱼的分类}(Duda教材经典例子):区分“三文鱼 (Salmon)”和“鲈鱼 (Sea Bass)”。单一特征(如长度)往往重叠严重,需要多特征(如长度+光泽度)组合以寻找最佳决策边界。
    \end{itemize}
\end{itemize}

\subsection{模式识别系统流程}
一个典型的模式识别系统包含以下五个核心环节:
\begin{enumerate}
    \item \textbf{数据获取 (Data Acquisition)}:通过传感器将物理变量转换为数字信号(如摄像头采集图像、麦克风采集声音)。
    \item \textbf{预处理 (Preprocessing)}:去噪、分割、平滑、归一化,旨在提高信噪比,简化后续处理。
    \item \textbf{特征提取与选择 (Feature Extraction \& Selection)}:
    \begin{itemize}
        \item \textbf{提取}:从原始数据中计算新的特征(变换)。
        \item \textbf{选择}:从一组特征中挑选出最具区分力的子集(降维),避免“维数灾难”。
    \end{itemize}
    \item \textbf{分类决策 (Classification/Decision)}:利用训练好的模型(分类器)对输入特征进行类别判定。
    \item \textbf{后处理 (Post-processing)}:利用上下文信息(Context)、先验知识或代价函数对分类结果进行修正(例如在OCR中利用语言模型纠错)。
\end{enumerate}

\subsection{模式识别系统设计}
设计周期通常是一个不断迭代的反馈过程:
\begin{itemize}
    \item \textbf{数据收集}:划分训练集(Training Set)、验证集(Validation Set)和测试集(Test Set)。
    \item \textbf{特征选择}:寻找具有\textbf{不变性 (Invariance)}的特征(如对平移、旋转、缩放不敏感)。
    \item \textbf{模型选择}:决定使用统计模型、神经网络模型还是句法模型;决定模型的复杂度。
    \item \textbf{训练 (Training)}:利用样本进行学习,估计模型参数。
    \item \textbf{评价 (Evaluation)}:核心指标是\textbf{泛化能力 (Generalization)},即在未见过的测试数据上的表现,需警惕\textbf{过拟合 (Overfitting)}。
\end{itemize}

\subsection{模式识别方法分类}
\begin{description}
    \item[基于数据形式]:
    \begin{itemize}
        \item \textbf{统计模式识别 (Statistical PR)}:基于特征向量,利用概率论和统计学理论(\textbf{本课程重点})。
        \item \textbf{句法/结构模式识别 (Syntactic/Structural PR)}:将模式分解为基元(Primitives),利用形式语言和语法规则描述结构关系(适合汉字、染色体等结构性强的对象)。
    \end{itemize}
    \item[基于学习方式]:
    \begin{itemize}
        \item \textbf{监督学习 (Supervised Learning)}:训练数据带有类别标签(Label)。
        \item \textbf{无监督学习 (Unsupervised Learning)}:训练数据无标签,旨在发现数据内部结构(如聚类)。
        \item \textbf{强化学习 (Reinforcement Learning)}:通过与环境交互,基于奖励/惩罚机制进行学习。
    \end{itemize}
\end{description}

\subsection{本课程内容体系}
\begin{itemize}
    \item \textbf{基础}:贝叶斯决策理论(统计模式识别的基石)。
    \item \textbf{估计}:概率密度函数估计(参数法与非参数法)。
    \item \textbf{分类器}:线性分类器、非线性分类器(神经网络、SVM、决策树)。
    \item \textbf{工程}:特征提取与选择、聚类分析。
\end{itemize}

\newpage
\section{贝叶斯决策理论 \small{(3学时 向世明)}}

\subsection{核心公式}
首先,我们需要充分理解先验、似然、后验的意义。在贝叶斯决策理论中,最重要的公式是贝叶斯公式:
\begin{equation}
    P(\omega_i|x) = \frac{p(x|\omega_i)P(\omega_i)}{p(x)}
\end{equation}

其中,$P(\omega_i)$ 是类别 $\omega_i$ 的先验概率,$p(x|\omega_i)$ 是在类别 $\omega_i$ 下观测到样本 $x$ 的似然概率密度,$P(\omega_i|x)$ 是在观测到样本 $x$ 后类别 $\omega_i$ 的后验概率。

一个典型案例是HIV检测,假设某种检测方法对HIV阳性患者的检测准确率为99.9\%,对HIV阴性患者的检测准确率为99.5\%。如果一个人被检测为阳性,我们需要计算他实际患有HIV的概率。假设总体中HIV阳性率为0.1\%,则可以利用贝叶斯公式计算后验概率。
概率计算为:
\begin{equation}
    P(\text{HIV}^+| \text{Test}^+) = \frac{P(\text{Test}^+|\text{HIV}^+)P(\text{HIV}^+)}{P(\text{Test}^+)}
\end{equation}

\subsection{最小错误率贝叶斯决策}
等价于最大后验概率决策:
\begin{equation}
    \text{Decide } \omega_i \text{ if } P(\omega_i|x) > P(\omega_j|x), \forall j \neq i
\end{equation}

这比较容易理解,因为大多数情况下我们总是希望在当前特征下做出最有可能的决策。

\subsection{最小风险贝叶斯决策}
在实际应用中,不同类型的错误造成的后果严重程度不同(例如误诊癌症),因此引入损失函数(Loss Function)的概念。
定义 $\lambda(\alpha_i|\omega_j)$ 为当真实类别为 $\omega_j$ 时采取决策 $\alpha_i$ 所带来的损失。

我们的目标是最小化\textbf{期望风险(Expected Risk)}。对于观测样本 $x$,采取决策 $\alpha_i$ 的\textbf{条件风险}定义为:
\begin{equation}
   E_{\omega|x}[\lambda(\alpha_i|\omega)] = \sum_{j=1}^{c} \lambda(\alpha_i|\omega_j) \underbrace{P(\omega_j|x)}_{\text{在该条件下}\omega\text{发生的概率}}
\end{equation}

\textbf{决策规则:}
选择风险最小的决策 $a$:
\begin{equation}
    a = \arg \min_{j=1,\dots,a} R(\alpha_j|x)
\end{equation}

\textbf{特例:0-1 损失函数}
如果损失函数定义为“对则无损,错则罚1”:
\begin{equation}
    \lambda(\alpha_i|\omega_j) = \begin{cases} 0, & i=j \\ 1, & i \neq j \end{cases}
\end{equation}

此时,最小风险决策完全等价于\textbf{最小错误率贝叶斯决策}(即最大后验概率决策 MAP)。

\subsection{分类器设计}
分类器可以看作是一个计算一组\textbf{判别函数(Discriminant Functions)} $g_i(x)$ 并选择最大值的机器。

\textbf{决策规则:}
如果 $g_i(x) > g_j(x), \forall j \neq i$,则将 $x$ 归类为 $\omega_i$。

\textbf{判别函数的形式:}
\begin{enumerate}
    \item \textbf{基础形式:}
    常用的判别函数形式有:
    \begin{itemize}
        \item $g_i(x) = P(\omega_i|x)$ (直接使用后验概率)
        \item $g_i(x) = p(x|\omega_i)P(\omega_i)$ (去掉分母 $p(x)$)
        \item $g_i(x) = \ln p(x|\omega_i) + \ln P(\omega_i)$ (取对数,变乘法为加法)
    \end{itemize}

    \item \textbf{更一般形式 (General Case):}
    PPT中给出了判别函数的更一般定义:
    \begin{equation}
        g_i(x) = f(p(x|\omega_i)) + h(x, \omega_i)
    \end{equation}
    其中:
    \begin{itemize}
        \item $f(\cdot)$ 是单调函数(通常取 $\ln$),用于变换似然度。
        \item $h(x, \omega_i)$ 是包含先验概率 $P(\omega_i)$ 和损失 $\lambda$ 的项。
    \end{itemize}

    \item \textbf{注:与神经网络的联系}
    这与现代神经网络最后的\textbf{判别器 (Discriminator)} 设计本质一致。神经网络最后一层(Logits)输出的值 $z_i = w_i^T x + b_i$,实际上就是在拟合这个 $g_i(x)$。其中 $w_i^T x$ 对应特征的似然度部分,而偏置 $b_i$ 往往隐含了先验概率 $P(\omega_i)$ 的信息。
\end{enumerate}

\subsection{高斯密度下的判别函数}
当类条件概率密度 $p(x|\omega_i)$ 服从多元正态分布 $N(\mu_i, \Sigma_i)$ 时,判别函数具有解析解。

多元正态分布概率密度公式:
\begin{equation}
    p(x) = \frac{1}{(2\pi)^{d/2}|\Sigma|^{1/2}} \exp \left( -\frac{1}{2}(x-\mu)^T \Sigma^{-1} (x-\mu) \right)
\end{equation}

取对数形式的判别函数 $g_i(x)$ 为:
\begin{equation}
    g_i(x) = -\frac{1}{2}(x-\mu_i)^T \Sigma_i^{-1} (x-\mu_i) - \frac{1}{2}\ln|\Sigma_i| + \ln P(\omega_i)
\end{equation}

根据协方差矩阵 $\Sigma_i$ 的不同,分为三种情形:

\subsubsection{情形一:$\Sigma_i = \sigma^2 I$(各特征独立且方差相等)}
此时各类协方差矩阵相同且为对角阵,几何上数据分布呈正球体,大小相同。判别函数可简化。

\textbf{1. 先验概率相等 ($P(\omega_i) = P(\omega_j)$):}
\begin{itemize}
    \item \textbf{判别依据:} 此时 $\ln P(\omega_i)$ 项为常数可忽略,决策规则简化为最小化\textbf{欧氏距离平方}:
    \[ g_i(x) \propto -\|x - \mu_i\|^2 \]
    \item \textbf{分类器名称:} \textbf{最小距离分类器 (Minimum Distance Classifier)}。
    \item \textbf{决策边界:} 连接两类均值向量 $\mu_i$ 和 $\mu_j$ 线段的\textbf{垂直平分面},经过中点 $x_0 = \frac{1}{2}(\mu_i + \mu_j)$。
\end{itemize}

\textbf{2. 先验概率不相等 ($P(\omega_i) \neq P(\omega_j)$):}
\begin{itemize}
    \item \textbf{判别依据:} 依然是线性判别函数,但包含偏置项:
    \[ g_i(x) = \frac{1}{\sigma^2}\mu_i^T x - \frac{1}{2\sigma^2}\mu_i^T \mu_i + \ln P(\omega_i) \]
    \item \textbf{决策边界:} 决策面依然与 $\mu_i - \mu_j$ \textbf{正交},但\textbf{不再经过连线中点}。
    \item \textbf{偏移规律:} 决策面会向先验概率较小的一侧偏移(即先验概率大的类别占据更大的决策区域)。偏移量为:
    \[ x_0 = \frac{1}{2}(\mu_i + \mu_j) - \frac{\sigma^2}{\|\mu_i - \mu_j\|^2} \ln \frac{P(\omega_i)}{P(\omega_j)} (\mu_i - \mu_j) \]
\end{itemize}

\subsubsection{情形二:$\Sigma_i = \Sigma$(协方差矩阵相等)}
此时各类数据分布呈超椭球体,形状和方向相同(平行),但中心不同。

\textbf{1. 先验概率相等 ($P(\omega_i) = P(\omega_j)$):}
\begin{itemize}
    \item \textbf{判别依据:} 最小化\textbf{马氏距离 (Mahalanobis Distance)} 平方:
    \begin{equation}
        r^2 = (x-\mu_i)^T \Sigma^{-1} (x-\mu_i)
    \end{equation}
    \item \textbf{决策边界:} 经过均值连线中点的超平面。但由于 $\Sigma$ 不再是单位阵,决策面通常\textbf{不与均值连线正交}。
\end{itemize}

\textbf{2. 先验概率不相等 ($P(\omega_i) \neq P(\omega_j)$):}
\begin{itemize}
    \item \textbf{判别依据:} 线性判别函数 $g_i(x) = w_i^T x + w_{i0}$,其中 $w_i = \Sigma^{-1}\mu_i$。
    \item \textbf{决策边界:} 决策面 $x_0$ 同样会偏离中点,向先验概率较小的一方偏移。
\end{itemize}

\subsubsection{情形三:$\Sigma_i \neq \Sigma_j$(协方差矩阵任意)}
\begin{itemize}
    \item \textbf{几何意义:} 各类数据的分布形状、方向、大小都不一样。
    \item \textbf{判别函数:} 无法消除二次项,形式为 $x^T W_i x + w_i^T x + w_{i0}$,是\textbf{二次判别函数}。
    \item \textbf{决策边界:} \textbf{超二次曲面}。根据协方差矩阵的差异,可能表现为超球面、超椭球面、超双曲面或超平面等多种形态。
\end{itemize}

\subsection{错误率分析}
\textbf{贝叶斯错误率(Bayes Error)}是理论上分类器能达到的最低错误率。
\begin{equation}
    P(\text{error}) = \int P(\text{error}|x)p(x)dx
\end{equation}
对于两类问题,错误率等于两个类条件概率密度曲线重叠部分的面积。特征的分离度越好,重叠越小,贝叶斯错误率越低。



\newpage
\section{概率密度函数估计 \small{(6学时 张燕明)}}
\subsection{基本概念}
\subsection{最大似然估计}
\subsection{贝叶斯估计}
\subsection{正态分布下的贝叶斯估计}
\subsection{贝叶斯学习}
\subsection{特征维数问题}
\subsection{期望最大法(EM)}
\subsection{隐马尔可夫模型}

\newpage
\section{非参数法 \small{(3学时 张燕明)}}
\subsection{密度估计}
\subsection{Parzen窗方法}
\subsection{K近邻估计}
\subsection{最近邻分类器}
\subsection{K近邻分类器的改进}
\subsection{距离度量}

\newpage
\section{线性分类器设计 \small{(3学时 张燕明)}}
\subsection{引言}
\subsection{线性判别函数与决策面}
\subsection{广义线性判别函数}
\subsection{感知准则函数}
\subsection{松驰方法}
\subsection{线性最小二乘方法}
\subsection{多类线性判别函数}

\newpage
\section{神经网络和深度学习 \small{(9学时 张燕明)}}
\subsection{人工神经网络发展历程}
\subsection{人工神经网络基础}
\subsection{单层前馈神经网络}
\subsection{多层感知器与误差反向传播算法}
\subsection{BP算法讨论}
\subsection{反馈神经网络}
\subsection{径向基函数网络}
\subsection{自组织映射}
\subsection{深度学习简介}
\subsection{波尔兹曼机简介}
\subsection{自编码器}
\subsection{卷积神经网络}
\subsection{Recurrent Neural Network}

\newpage
\section{特征提取与选择 \small{(6学时 张燕明)}}
\subsection{维数灾难}
\subsection{特征提取}
\subsection{线性方法}
\subsection{非线性方法}
\subsection{特征选择}

\newpage
\section{模型选择 \small{(3学时 张燕明)}}
\subsection{模型选择原则}
\subsection{模型评价标准}
\subsection{基于偏差和方差的分类器设计准则}
\subsection{分类器集成}
\subsection{Adaboost学习方法及其应用}

\newpage
\section{聚类分析 \small{(6学时 张燕明)}}
\subsection{K均值聚类}
\subsection{模糊K均值聚类}
\subsection{层次聚类}
\subsection{谱聚类}
\subsection{在线聚类}
\subsection{集成聚类}

\newpage
\section{支持向量机与核方法 \small{(6学时 向世明)}}
\subsection{结构风险、经验风险与VC维}
\subsection{线性可分支持向量机与硬间隔最大化}
\subsection{线性支持向量机与软间隔最大化}
\subsection{非线性支持向量机}
\subsection{支持向量机的优化与扩展}

\newpage
\section{决策树方法 \small{(3学时 向世明)}}
\subsection{决策树模型、学习与特征选择}
\subsection{决策树生成与剪枝(ID3, C4.5, CART等算法)}
\subsection{随机森林算法}

\newpage
\section{模式识别前沿趋势 \small{(3学时 向世明)}}
\subsection{课程内容总体回顾}
\subsection{模式识别发展现状与新动态}
\subsection{模式识别中的挑战性问题}

\newpage
\section{考核 \small{(6学时 向世明)}}
\subsection{考试+答疑}

\newpage
%===========================================================
% 常用 LaTeX 模板区
%===========================================================
\appendix
\section*{附录:模式识别常用 LaTeX 模板}
\addcontentsline{toc}{section}{附录:常用 LaTeX 模板}

\subsection*{1. 学术三线表模板}
\begin{table}[H]
    \centering
    \caption{不同分类器在 Iris 数据集上的错误率对比}
    \label{tab:pr_error}
    \begin{tabular}{lccc} 
        \toprule
        \textbf{算法} & \textbf{训练集错误率} & \textbf{测试集错误率} & \textbf{计算耗时(ms)} \\
        \midrule
        KNN ($K=3$) & 0.02 & 0.04 & 12 \\
        Fisher LDA & 0.05 & 0.06 & \textbf{2} \\
        SVM (RBF Kernel) & \textbf{0.01} & \textbf{0.03} & 45 \\
        \bottomrule
    \end{tabular}
\end{table}

\subsection*{2. 图片插入模板 (已注释)}
% \begin{figure}[H]
%     \centering
%     \includegraphics[width=0.7\textwidth]{figures/decision_boundary.png} 
%     \caption{线性分类器与非线性分类器的决策边界对比}
%     \label{fig:boundary}
% \end{figure}

\subsection*{3. 模式识别核心公式模板}

\paragraph{贝叶斯决策规则:}
后验概率计算:
$$ P(\omega_i | \bm{x}) = \frac{p(\bm{x} | \omega_i) P(\omega_i)}{p(\bm{x})} $$
最小错误率决策:若 $P(\omega_i|\bm{x}) > P(\omega_j|\bm{x})$,则判定 $\bm{x} \in \omega_i$。

\paragraph{多元正态分布 (Gaussian Density):}
$$ p(\bm{x}) = \frac{1}{(2\pi)^{d/2} |\bm{\Sigma}|^{1/2}} \exp \left[ -\frac{1}{2} (\bm{x} - \bm{\mu})^T \bm{\Sigma}^{-1} (\bm{x} - \bm{\mu}) \right] $$

\paragraph{线性判别函数 (Linear Discriminant):}
$$ g(\bm{x}) = \bm{w}^T \bm{x} + w_0 $$

\paragraph{最大似然估计 (MLE):}
$$ \hat{\bm{\theta}} = \arg \max_{\bm{\theta}} \prod_{k=1}^{N} p(\bm{x}_k | \bm{\theta}) \quad \Rightarrow \quad \hat{\bm{\theta}} = \arg \max_{\bm{\theta}} \sum_{k=1}^{N} \ln p(\bm{x}_k | \bm{\theta}) $$

\paragraph{SVM 优化目标 (硬间隔):}
$$ \min_{\bm{w}, b} \frac{1}{2} ||\bm{w}||^2 \quad \text{s.t.} \quad y_i(\bm{w}^T \bm{x}_i + b) \ge 1, \quad i=1,\dots,N $$

\paragraph{特征值分解 (PCA):}
协方差矩阵 $\bm{S}$ 的特征方程:
$$ \bm{S} \bm{u}_i = \lambda_i \bm{u}_i $$

\paragraph{K-Means 目标函数:}
$$ J = \sum_{j=1}^{K} \sum_{\bm{x} \in C_j} ||\bm{x} - \bm{m}_j||^2 $$

\end{document}